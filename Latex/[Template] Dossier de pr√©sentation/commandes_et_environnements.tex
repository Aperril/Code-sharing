\chapter{Commandes et environnements}
\minitoc

\newpage
\section{Crédits}

\url{http://web.mat.bham.ac.uk/R.W.Kaye/latex/defining/defining.pdf}
\url{https://tex.stackexchange.com/questions/16814/restrict-macro-to-an-environment}
\url{https://github.com/mathnaudint}

\section{Commandes}

Template de la définition d'une commande :
\begin{verbatim}
\newcommand{\cmd_name}[nb_args]{ici se trouve le contenu de la commande
et #2 et #1 sont les valeurs des arguments si nb_args >= 2}
\end{verbatim}


\newcommand{\seq}[2][n]{#2_{1},\ldots,#2_{#1}}
Définition de la commande :
\begin{verbatim}
\newcommand{\seq}[2][n]{#2_{1},\ldots,#2_{#1}}
\end{verbatim}

La commande s'appelle avec \verb+\seq+ et possède deux arguments notés \verb+#1+ et \verb+#2+. Elle permet d'afficher un $n$-uplet (paramètre optionel \verb+[n]+).
Utilisation de la commande :
\begin{verbatim}
For some \(i \in \{ \seq{t} \} \) we have \( f(i)=\seq[m]{\alpha} \)
\end{verbatim}
donne :\\
For some \(i \in \{ \seq{t} \} \) we have \( f(i)=\seq[m]{\alpha} \)

\bigskip
\par
Les commandes peuvent être définies dans un environnement et celle-ci sont locales à l'environnement.

\section{Environnements}

\newenvironment{qsi}[1]%
{\begin{quote}#1 wrote,\begin{sloppypar}\it}%
{\end{sloppypar}\end{quote}}



\newenvironment{quote2}[1]%
{
  \begin{quote}#1 wrote,\begin{sloppypar}\it
  \newcommand{\gf}[1]{\mathbb{F}_{##1}}
}%
{
  \end{sloppypar}\end{quote}
}


\bigskip
\begin{verbatim}
\newenvironment{qsi}[1]%
{\begin{quote}#1 wrote,\begin{sloppypar}\it}%
{\end{sloppypar}\end{quote}}


\begin{qsi}{Pascal}
I love finite fields !
\end{qsi}
\end{verbatim}
donne :\\
\begin{qsi}{Pascal}
I love finite fields !
\end{qsi}




\begin{verbatim}
\newenvironment{quote2}[1]%
{
  \begin{quote}#1 wrote,\begin{sloppypar}\it
  \newcommand{\gf}[1]{\mathbb{F}_{##1}}
}%
{
  \end{sloppypar}\end{quote}
}

\begin{quote2}{Pascal}
I love finite fields ! Let's play with $\gf{2^m}$
\end{quote2}
\end{verbatim}
donne :\\
\begin{quote2}{Pascal}
I love finite fields ! Let's play with $\gf{2^m}$
\end{quote2}


Définir une commande dans un environnement nécessite un double \#\# lors de l'appel de l'argument de la commande dans l'environnement.

\section{Tests des environnements}

\begin{verbatim}
\newenvironment{mymathenv}%
{
  % liste des commandes dans l'environnement
  \newcommand{\gf}[1]{\mathbb{F}_{##1}}
  \newcommand{\gftwo}[1]{\mathbb{F}_{2^{##1}}}
  \newcommand{\gfq}[1]{\mathbb{F}_{q^{##1}}}

  \newcommand{\x}{\mathbf{x}}
  \newcommand{\mbf}[1]{\mathbf{##1}}
  \newcommand{\vect}[1]{##1_1, \ldots, ##1_n}
}%
{}



\begin{watermarking}
Soit $\mbf{a} \in \gftwo{m}$. Soit $\x = (\vect{x}), \mbf{y} \in \mathbb{R}^n$.
\end{watermarking}
\end{verbatim}
donne :\\
\begin{watermarking}
Soit $\mbf{a} \in \gftwo{m}$. Soit $\x = (\vect{x}), \mbf{y} \in \mathbb{R}^n$.
\end{watermarking}