%------------------ECRIRE EN FRANCAIS-----------
\setlength{\parindent}{0cm}
\usepackage[frenchb]{babel}
\usepackage[T1]{fontenc} %permet les accents
\usepackage[cyr]{aeguill} %ajout guillemets spéciaux
\usepackage{lmodern}
%\usepackage[utf8]{inputenc} 
\usepackage[latin1]{inputenc}

%------------------MATHS---------------
\usepackage{amsmath, amsthm, amssymb} 
\usepackage{dsfont}
\usepackage{mathrsfs} %permet les jolies lettres rondes

%----------AJOUT COULEURS-----------------
\usepackage[svgnames]{xcolor}
\definecolor{dkgreen}{rgb}{0,0.6,0}
\definecolor{gray}{rgb}{0.5,0.5,0.5}
\definecolor{mauve}{rgb}{0.58,0,0.82}
\definecolor{gris}{gray}{0.9}

%-----------FIGURES ET SOUS-FIGURES-------------
\usepackage[pdftex]{graphicx}
\usepackage{float} %sous-figures
\usepackage{subfig} %sous-figures
\usepackage[rightcaption]{sidecap} %titre a droite

%--------------STYLISATION DES LIENS ET CITE--------
\usepackage{hyperref}
\hypersetup{ %Recolarisation
colorlinks=true,           
breaklinks=true,
urlcolor= blue, 
linkcolor= black,
citecolor= blue
}

%----------------ENUMERATIONS-----------
\usepackage{enumerate} %permet les énumérations

%-------------------ESPACES-------------
\newcommand{\N}{\mathbb{N}}
\newcommand{\Z}{\mathbb{Z}}
\newcommand{\R}{\mathbb{R}}
\newcommand{\C}{\mathbb{C}}

%-----------------NOUVELLES COMMANDES----------------
\newcommand{\fnct}[5]{#1: \begin{array}{ccc} #2 \longrightarrow &#3\\ #4&\longmapsto &#5\end{array}} %fonctions
\newcommand{\fnc}[5]{ \begin{array}{c|ccc} #1: &#2&\longrightarrow &#3\\ &#4&\longmapsto &#5\end{array}} %fonctions
\newcommand{\di}{\mbox{d}} 
\newcommand*{\norm}[1]{\left\lVert#1\right\rVert} % La norme
\newcommand*{\abs}[1]{\left \vert#1 \right \vert} %valeur absolue
\newenvironment{pushright}{
\begin{itemize}\item[\hspace{12pt}]}{\end{itemize}}
\newcommand{\HRule}{\rule{\linewidth}{0.5mm}} 

%---------------Tableau---------------
\usepackage{array,multirow,makecell}
\setcellgapes{1pt}
\makegapedcells
\newcolumntype{R}[1]{>{\raggedleft\arraybackslash }b{#1}}
\newcolumntype{L}[1]{>{\raggedright\arraybackslash }b{#1}}
\newcolumntype{C}[1]{>{\centering\arraybackslash }b{#1}}

%---------RECTIFICATION DES MARGES ETC-------
\topmargin-1.5cm
\textheight25cm
\textwidth17.6cm
\oddsidemargin-1cm
\evensidemargin0cm
\pagestyle{plain}
\setlength{\parindent}{0cm}

\newcommand{\reporttitle}{Rapport d'avancement}     % Titre
\newcommand{\reportauthor}{ } % Auteur